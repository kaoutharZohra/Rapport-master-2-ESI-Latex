% Chapter Template
\chapter*{Abstract}% Main chapter title
\addtotoc{Abstract}
\label{Chapter3} % Change X to a consecutive number; for referencing this chapter elsewhere, use \ref{ChapterX}

\lhead{\emph{Human detection and social navigation for a guide robot}}
\tab Currently, we are experiencing a great advancement in new technologies, especially in mobile robotics and service robotics. Mobile robots have invaded the human space, they share it with them, which obligates them to manage their interactions with people so they can integrate into human society.  And, to accomplish the different tasks or services offered by social robots to human, the mobile robotics, uses different research domains. One of the most exploited domain of service robotics is perception, particularly the detection of objects by means of different sensors. The nature of the social robot imposes on it an interaction with humans, therefore he has a strong need to detect them and distinguish them from other objects. Social robots interfere generally in populated environments; thus, it requires them to respect social aspect of humans and follow a set of social conventions. \vspace{5px}\\
\tab This report consists of the design and implementation of a person detection and social navigation for a guide robot. By using face and skeletal detection algorithms that function on Kinect vision and other leg sensors that use the laser sensor data, we present a design that enhances the detection module for a guide robot. Once the detection module provides satisfactory results, we exploit them in the social navigation of the robot, taking care of the respect to a set of rules and social conventions. These conventions are inspired by the theory of proxemic and sharing of social space around a person. Our system is tested on a simulator that demonstrates its proper functioning, it then is examined on the actual robotic platform to confirm the results of the simulation.\vspace{5px}\\
\tab \textbf{Key words: } mobile robotics, service robotics, interaction, social robot, detection of objects, sensors, social aspect of humans, social conventions, person detection, social navigation, Kinect, laser, theory of proxemic.
