% Chapter Template
\chapter*{Résumé}% Main chapter title

\label{Chapter2} % Change X to a consecutive number; for referencing this chapter elsewhere, use \ref{ChapterX}

\lhead{\emph{Les anti-patrons linguidtiques}}



Le développement logiciel est une tâche itérative et le code source est revu plusieurs fois par ceux qui l'ont écrit ou par d'autres développeurs.
Afin d'améliorer sa qualité et faciliter sa maintenance en minimisant son coût et son temps, pour ces raisons, le code source doit être compréhensible, qui dit compréhensibilité dit qualité de lexique utilisé dans le code.\vspace{5px}\\
En plus des métriques structurales utilisées pour mesurer la qualité du code source, il existe d'autres moyens, comme les lexicon bad smells et les anti-patrons linguistiques qui agissent sur le lexique du code source, les termes des identificateurs utilisés et des commentaires.
\vspace{5px}\\
Dans ce rapport nous synthétisons les différents travaux cités dans la littérature concernant les lexicon bad smells, les anti-patrons linguistiques et leur détection, et nous donnons nos critiques sur ces travaux.\vspace{5px}\\
\tab \textbf{Les mots clés : } anti-patrons linguistique, lexicon bad smell, patrons de conception, anti-patrons de conception, détection d'anti-patrons linguistiques, détection de lexicon bad smells.

