\section{Conclusion}
\tab Dans ce rapport, nous avons abordé le sujet des lexicon bad smells et d'anti-patrons linguistiques comme axe de recherches récent dans le domaine de la détection des patrons et des anti-patrons en général. \\ \tab

Cet axe n’est pas encore bien exploité, chose observée pendant la recherches bibliographique, les informations linguistiques ne sont pas toutes exploitées et  peu sont les algorithmes de détection, par exemple l'article \cite{karwin2010sql} propose quelques anti-patrons pour le langage SQL mais les anti-patrons linguistiques SQL ne sont pas invoqués, il n'ya pas de vrai catalogue pour les lexicon bad smells et d'anti-patrons linguistiques comme celui des patrons de concèption.
\\


\tab Nous avons établi la liaison entres l’information linguistique,et ses differents métriques et un bon code source,nous avons aussi touché à l’importance de détéction des lexicon bad smells et des anti-patrons linguistiques et leurs solutions refactorisées pour améliorer la qualité logicielle. Comme perspective, il  reste à définir un modèle d'anti-patrons lingustiques, établir une relation entre les lexicon bad smells et les anti-patrons linguistiques, pour enfin atteindre le but de réaliser un outil pour la détection automatique et exacte des anti-patrons linguistiques avec la possibilité de correction et de refactoring automatique.
